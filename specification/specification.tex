\documentclass[a4paper,12pt]{article}

\usepackage[utf8]{inputenc}  % Karakterkódolás beállítása (UTF-8)
\usepackage[T1]{fontenc}      % Kimeneti karakterkészlet (T1)
\usepackage[hungarian]{babel} % Magyar nyelvi támogatás
\usepackage{amsmath}
\usepackage{graphicx}
\usepackage{hyperref}
\usepackage{listings}

\title{Game of Life Programozási Feladat Specifikáció}
\author{Gáspár Róbert \\ Neptun kód: K8FD5S}
\date{\today}

\begin{document}

\maketitle

\section*{Feladat rövid ismertetése}

A házi feladat keretében egy klasszikus \textit{Game of Life} szimulációt kell megvalósítani. A \textit{Game of Life} egy sejtautomata, amit John Conway matematikus alkotott meg 1970-ben. A játék egy kétdimenziós rácson játszódik, ahol minden cella két állapotban lehet: él vagy halott. Az idő múlásával a sejtek állapota az alábbi szabályok szerint változik:

\begin{itemize}
    \item \textbf{Túlélés}: Egy élő sejt továbbra is életben marad, ha 2 vagy 3 szomszédos élő sejt veszi körül.
    \item \textbf{Szaporodás}: Egy halott sejt életre kel, ha pontosan 3 szomszédos élő sejt veszi körül.
    \item \textbf{Halál}: Minden más esetben a sejtek meghalnak, azaz halottak maradnak.
\end{itemize}

A feladat során egy interaktív \textit{Swing} grafikus felhasználói felületet (GUI) kell készíteni, amely lehetőséget biztosít a felhasználónak a rács méretének kiválasztására, a szimuláció elindítására, megállítására, valamint a mentések kezelésére (mentés és betöltés). A program lehetőséget nyújt arra is, hogy a felhasználó fájlba mentse a játék állapotát JSON formátumban, majd később újra betöltse azt. A játék nézete dinamikusan nagyítható és kicsinyíthető lesz, illetve a grid elmozdítható lesz a képernyőn. A felhasználó menükön és billentyűzet használatán keresztül érheti el az összes funkciót, mint például új játék indítása, játék mentése, betöltése vagy a program kilépése.

\section*{Use-case-ek}

\begin{itemize}
    \item \textbf{Új játék indítása}: A felhasználó a menüben vagy egy kezdő képernyőn elindíthat egy új játékot, ahol megadhatja a grid méretét (pl. 10x10, 50x50 stb.). A grid inicializálása után a felhasználó kattintással élővé tehet vagy kiüríthet sejteket a kezdő állapot beállításához, majd elindíthatja a szimulációt.
    
    \item \textbf{Játék indítása/leállítása}: A felhasználó egy gomb vagy menüpont segítségével elindíthatja és megállíthatja a szimulációt. A szimuláció során a sejtek állapota folyamatosan változik az előbb említett szabályok alapján, és a képernyőn vizuálisan frissül.
    
    \item \textbf{Játék mentése}: A felhasználó a menüben vagy egy megfelelő gombbal elmentheti az aktuális játékállást egy JSON fájlba. A mentett fájl tartalmazza a rács méretét, a sejtek állapotát.
    
    \item \textbf{Játék betöltése}: A felhasználó a menüben kiválaszthat egy korábban elmentett JSON fájlt, amely betölti a játék előző állapotát, beleértve a rács méretét és a sejtek állapotát.
    
    \item \textbf{Nézet nagyítása/kicsinyítése és mozgatása}: A felhasználó dinamikusan nagyíthatja és kicsinyítheti a grid nézetét, illetve elmozdíthatja azt, hogy egy adott részletet megfigyelhessen. Ezt az egér görgőjével vagy gombok segítségével érheti el.
    
    \item \textbf{Kilépés}: A felhasználó a menü segítségével kiléphet a programból. Ha a játék nincs elmentve, akkor egy figyelmeztetés jelenik meg, hogy megerősítést kérjen a kilépésről.
\end{itemize}

\newpage

\section*{Megoldási ötlet vázlatos ismertetése}

A megoldás alapvetően a Java \textit{Swing} keretrendszerre épül, mivel a specifikáció megköveteli a Swing GUI használatát. A program főbb részei a következők:

\begin{itemize}
    \item \textbf{Grafikus felhasználói felület (GUI)}: A GUI menükből és panelekből épül fel. A menük biztosítják a felhasználó számára az alapvető funkciókat, mint a játék indítása, mentése, betöltése és kilépése. A grid nézete egy \textit{JPanel} lesz, amelyben az egyes sejtek állapotai jelennek meg. A grid vizuálisan is dinamikusan változik a játék során, a \textit{Graphics} osztály segítségével rajzolva.
    
    \item \textbf{Rács nézet és transformációk}: A rács nézete nagyítható és kicsinyíthető, valamint elmozdítható lesz a képernyőn. Ehhez a \textit{Graphics2D} osztály és az affine transzformációk lesznek alkalmazva, hogy a grid mérete és pozíciója dinamikusan módosítható legyen.
    
    \item \textbf{Gyűjtemény keretrendszer}: A grid belső adatstruktúrájának kezeléséhez egy kétdimenziós tömböt használunk, amelyben a sejtek állapotát tároljuk.
    
    \item \textbf{JSON fájlkezelés}: A játékállások mentéséhez és betöltéséhez a \textit{Jackson} vagy \textit{Gson} könyvtárakat használjuk, amelyek lehetővé teszik az objektumok JSON formátumban történő szerializálását és deszerializálását. A grid állapota és a játék további adatai egy fájlban lesznek eltárolva, amit a felhasználó betölthet később.
    
    \item \textbf{Tesztelés}: A program működésének ellenőrzésére \textit{JUnit} tesztelési keretrendszer kerül alkalmazásra. Legalább három osztály és tíz metódus kerül tesztelésre.
\end{itemize}

\end{document}
